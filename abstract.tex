%\abstracttitle
% Single spacing can be turned on for the abstract
%

%The extraction of the report not the research.
%
%The aim, objectives and motivations are... 
%
%What have been done...
%
%The results of experiment...
%
%One sentence about the conclusions and future plan.
%About half page long.

%To explore how the brain may recognise objects in its general,accurate and energy-efficient manner, this paper proposes the use of a neuromorphic hardware system formed from a Dynamic Video Sensor~(DVS)
%silicon retina in concert with the SpiNNaker real-time Spiking Neural
%Network~(SNN) simulator.
%As a first step in the exploration on this platform a recognition system for dynamic hand postures is developed, enabling the study of the methods used in the visual pathways of the brain.
%Inspired by the behaviours of the primary visual cortex, Convolutional Neural Networks (CNNs) are modelled using both linear perceptrons and spiking Leaky Integrate-and-Fire (LIF) neurons.
%
%In this study's largest configuration using these approaches, a network of 74,210 neurons and 15,216,512 synapses is created and operated in real-time using 290 SpiNNaker processor cores in parallel and with 93.0\% accuracy.
%A smaller network using only 1/10th of the resources is also created,
%again operating in real-time, and it is able to recognise the postures
%with an accuracy of around 86.4\% - only 6.6\% lower than the much larger system.
%The recognition rate of the smaller network developed on this neuromorphic system is sufficient for a successful hand posture recognition system, and demonstrates a much improved cost to performance trade-off in its approach.

The human brain recognises a huge number of objects rapidly with ease even in cluttered and natural scenes.
This robust object recognition of the biological system is invariant to the change of position, scale, viewing angle, etc. (known as transformation invariance).
While the major stumbling problem of computer object recognition lies in the poor robustness to these transformations.

Exploring and mimicking invariant object recognition within the brain is a promising approach to tackling the computational difficulty;
in turn it also contributes to understanding biological visual processing by means of mimicking neural activity in the visual system of the brain.

Thanks to its high-performance massively parallel processing, SpiNNaker makes it possible to simulate large-scale neural networks in real-time.
Thus, to explore the visual pathway of primate brain, this work exploits a neuromorphic hardware system formed from a Dynamic Video Sensor~(DVS) silicon retina in concert with SpiNNaker.
As a first step in the exploration, a recognition system for dynamic hand postures is developed on the platform.
Inspired by the behaviour of the primary visual cortex, Convolutional Neural Networks (CNNs) are modelled using both linear perceptrons and Leaky Integrate-and-Fire (LIF) neurons.

Future work is proposed to build an object recognition system with position, scale and view invariance by modelling the hierarchical visual pathway up to the inferotemporal (IT) cortex.
The proposed system will be able to recognise 200 objects in real time exploiting LIF neurons.

