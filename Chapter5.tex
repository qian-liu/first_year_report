\chapter{Contributions and Research Plan}
\label{cha:plan}
\section{Contributions}
%Motivation
To explore how brain may recognise objects in its general, accurate, invariant and energy-efficient manner, this work proposes the use of a neuromorphic hardware system which includes a DVS retina connected to SpiNNaker, a real-time SNN simulator.
%Problem
Building a hand gesture recognition system based on this bespoke hardware for dynamic hand postures is a first step in the study of the ventral visual pathway in the brain.
%Methods
Inspired by the structures of the primary visual cortex, convolutional neural networks are modelled using both linear perceptrons and LIF neurons as V1-liked neurons.
This model is position invariant to recognise moving postures.

%Results
%The larger network of 74,210 neurons and 15,216,512 synapses runs smoothly in real-time on SpiNNaker using 290 cores within a 48-node board.
%The smaller network using 1/10 of the resources is able to recognise the postures in real-time with an accuracy about 86.4\% in average, which is 
%only 6.6\% lower than the former but with a better cost/performance ratio.

The detailed contributions are listed below:
\begin{itemize}
	\item modelled a convolutional neural network to recognise moving postures (position invariant) with V1-liked neurons.
	\item translated the conventional artificial neural networks of perceptrons to spiking neural networks of LIF neurons by Siegert function.
	\item configured the neuromorphic platform to communicate with the retina to perform real-time posture recognition using spiking neurons.
	\item maintain the software to make SpiNNaker receive correct recorded spikes to compare the performance with perceptrons and visualise the live spikes to probe the neural activities in real time.
\end{itemize}
	
%In this paper, we implement a dynamic hand posture recognition system running completely on a hardware neuromorphic platform.
%The SNN based classifier is able to recognise moving hand postures instead of static digit or face recognition proposed before.
%The network model is translated from linear perceptrons to LIF spiking neurons with a 10\% drop of accuracy in 30~ms windowing;
%while the performance reaches and even exceeds the perceptrons version when the window length is set to 300~ms.
%Various network sizes are configured to explore the cost and performance trade-off.
%In the tests of linear perceptrons the recognition rate of the smaller network is \% lower for the template matching model and \% lower for the trained MLP model.
%For the SNN based real-time experiments, both the larger network with $74,320$ LIF neurons and $15,216,512$ synapses, and the smaller network with 1/10 of the neurons and 1/50 of the synapses run smoothly on SpiNNaker.
%The gap of recognition rate narrows when the spiking rate is sampled into wider frame of 300~ms.
%The recognition rates of both linear perceptrons (\%) and spiking neurons (\%) with 32 $\times$ 32 input resolution are adequate for the recognition of the moving hand postures.
%This work is a good attempt to start exploring the visual process of the brain.

%Later, we will look more into the overall network latency, which could be due to the system latency and the performance of the rate coding. 
\section{Publications}
\begin{itemize}

	\item Q. Liu and S. Furber, "Real-time recognition of dynamic hand posture on a neuromorphic system." Artificial Neural Networks ICANN 2015. Springer Berlin Heidelberg. (Under review)


	\item Q. Liu, X. Lagorce, E. Stromatias, D. Emmanouilidou, R. Benosman, S. Furber and S. Liu, "A large-scale, real-time sound localization on a neuromorphic platform." Neuromorphic Engineering. (Under proof reading of co-authors)

	\item Q. Liu, C. Patterson, S. Furber, Z. Huang, Y. Hou, and H. Zhang, "Modeling populations of spiking neurons for fine timing sound localization." in Neural Networks (IJCNN), The 2013 International Joint Conference on, pp. 1–8, Aug 2013.
\end{itemize}

\section{Future Work} 
\begin{figure}[h!]
	\centering
	\includegraphics[width=0.75\textwidth]{pics/3D_plan.pdf}
	\caption{3D representation of the research plan on the transformation-invariant object recognition system.
	Three milestones are highlighted with red stars indicating the expected targets of the object recognition networks.
	}
	\label{Fig:3Dplan}
\end{figure}
The proposed research plan is illustrated in Figure~\ref{Fig:3Dplan}.
To build a biologically-plausible object recognition system with spiking neurons, this work will be completed with different scopes in three stages.
The recognition ability of the system is measured in three dimensions: the hierarchy layers, degree of invariance and the network size.
 
This work will contribute to the understanding of biological visual processing by means of mimicking the neural activities in the ventral stream.
More importantly, the research will apply the accurate, rapid, robust and effortless approaches to artificial systems by exploring the brain's invariant object recognition.

The performance of the real-time recognition system will be tested on each milestone to validate the success of the models.
The neural activities and recognition rate will be compared with biological data.

The key research steps are listed in Figure~\ref{Fig:gantt}.
Since the increment work flow is hard to present in Gantt charts, only the work for the first milestone is drawn.
The subsequent sections will outline the key research stages.



\begin{figure}[h!]
	\centering
	\includegraphics[width=0.8\textwidth]{pics/gantt.pdf}
	\caption{Gantt chart of the work flow for the first milestone.
	The main research works are listed on the left.
	Different from the example, in the following work, not only achieving a milestone but also any increase in any dimension will result in tuning and benchmark testing.}
	\label{Fig:gantt}
\end{figure}
\subsection{Invariant Object Recognition}
As stated above the brain recognises huge amount of objects rapidly and effortlessly even in cluttered and natural scenes.
While the major stumbling crux of the computer object recognition systems lies in the invariance problem.
To explore the invariant object recognition of the brain in a biologically plausible way is the right place to solve the computational difficulty.
\subsubsection{Position Invariance}
Position invariance in the lower level of V1-liked neurons has been achieved in the preliminary work by convolving receptive fields with Gabor kernels.
The following work in accordance with Figure~\ref{Fig:3Dplan} will focus on how to expand the position invariance to higher hierarchical level of the ventral stream.
\subsubsection{Scale Invariance}
Similar to orientation detection, V1 provides overcomplete population re-representations of visual image on the features of scale, frequency and orientation.
It forms the basis of scale invariant object recognitions.
The complexity of this work will increase as the number of hierarchy layers grows.
\subsubsection{View Invariance}
A difficult specificity-invariance trade-off occurs in the view invariant recognition tasks, since the recogniser should be able to discriminate different objects while at the same time it is also tolerant to viewing angel transformations.
Learning will play a very important role in this work, where objects observed with multiple view points can be recognised even only single view point is trained.
\subsection{Modelling the Hierarchy Layers}
As the visual information conducts along the ventral stream, neurons become selective for increasingly complex features. 
Along with this growing complexity of the preferred stimulus, neurons become more and more tolerant to the exact position and scale of the stimulus within their receptive fields.
To satisfy the functional discoveries, this work will employ learning and compare with biological data.
\subsection{Size Scaling}
The milestones set for the dimension of size scaling is in accordance with experiments data.
In paper~\cite{hegde2004temporal}, the classical receptive field of the V2 cell consists of 48 grating stimuli and 80 contour stimuli; while Zoccolan et al.~\cite{zoccolan2007trade} tested the IT neurons with 213 images.

Thanks to the massive-parallel neural simulation of SpiNNaker system, to make a great real-time invariant object recognition becomes possible.
However, it also requires the software development to support huge neural networks.  
\subsection{Integration}
To reach the milestone of building an object recognition system with position, scale and view invariance, integration of these separate models will be a challenge.
It does not only require placing the models physically together, but also merge the functions.
As illustrated in~\ref{sec:orIT}, single neurons are tuned to different features and object identities.
This work asks for investigation on population coding and learning. 
\subsection{Tuning}
Tuning is the key to make the object recognition system success.
In the preliminary work, Siegert transformation function is used to adjust perceptral weights to spiking LIF neurons.
It is a strong backer to guarantee the feasibility of the work.
However, learning algorithms such of STDP of spiking neural networks are supposed to be employed to make the system more biologically plausible.
On the other hand, this work will provoke the learning algorithm study in SpiNNaker group.
\subsection{Benchmark Performance}
The performance of the real-time recognition system will be tested on each milestone to validate the success of the models.
The neural activities and recognition rate will be compared with biological data.
\subsubsection{Building Dataset}
Building a well-labelled retinal output dataset is essential in spike-based object recognition study.
Unified benchmarks with AER format will be ideal for SNN study, because of its non-frame, event-based fashion.
These benchmarks make it possible for research groups to test their SNN model without a silicon retina present.
It will boost the communication, comparison and collaboration in the community.
In addition, it requires a lively discussion and cooperation with neuroscientists, where data can be derived and tested.
\subsubsection{Testing/Comparing}
The testing and comparing on the dataset will verify the reliability of the models.
By comparing with the biological data, the model can be rectified and improved.
The more data it compares with, the closer it could untangle the object representation. 


%\subsection{Optional: Action Recognition}
%vision attention.
%short-term memory.
%\subsection{Optional: Sensor Fusion with Auditory Processing}
%platform.
%applications as lip-reading and speaker identification.

